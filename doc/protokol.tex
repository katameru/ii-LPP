\documentclass[a4paper, 10pt]{article}
\usepackage[MeX]{polski}
\usepackage[utf8]{inputenc}
\usepackage{amsmath,amsthm,amssymb}
\usepackage[lmargin=2.7cm]{geometry}
\newcommand{\newtitle}[1]{
\begin{center}
\thispagestyle{empty}
{\Large Licencjacka Pracownia Oprogramowania, zespół 21}\\[0.5cm]
{\Large Instytut Informatyki Uniwersytetu Wrocławskiego 2011/12}\\[2.5cm]
{\huge Dokumentacja projektu \textbf{Wtomigraj}}\\[0.5cm]
{\huge #1}\\[0.5cm]
{\large Marcin Januszkiewicz, Grzegorz Łoś}\\[0.5cm]
\vfill
{\large Wrocław, \today}
\end{center}
\newpage
}
\renewcommand{\thesection}{\arabic{section}.}
\renewcommand{\thesubsection}{\arabic{section}.\arabic{subsection}.}
\renewcommand{\thesubsubsection}{\arabic{section}.\arabic{subsection}.\arabic{subsubsection}.}


\begin{document}

%strona tytułowa
\newtitle{Komunikacja klient-serwer}
\tableofcontents

\newpage
\section{Opis protokołu}
Komunikacja pomiędzy klientem a serwerem odywa się przez wymianę pakietów UDP z wiadomościami zawartymi w polu danych. W warstwie aplikacji protokół ma formę pytanie-odpowiedź, co zapewnia odporność na błędy sieci związane zstratą pakietów. Wiadomości przesyłane są w formacie JSON, co pozwala na wykorzystanie istniejących bibliotek do prostego kodowania i dekodowania informacji.

Każdy obiekt JSON zawiera pola:
\begin{itemize}
 \item type -- typ komunikatu, obiekt typu String. Wszystkie typy komunikatów omawiamy poniżej.
 \item res -- pole boolowskie ustawione na true, gdy pakiet jest odpowiedzią, w przeciwnym razie jest to pytanie.
 \item id -- w przypadku pakietów-pytań jest to numer identyfikacyjny pytania, natomiast dla odpowiedzi oznacza identyfikator pytania, na które odpowiadamy.
\end{itemize}
Ponadto dla każdego typu komunikatu obiekt JSON może zawierać dodatkowe pola.

\section{Lista możliwych komunikatów}
\subsection{Komunikaty wysyłane przez klienta do serwera}
\begin{description}
 \item[newclient] -- komunikat informujący o chęci przyłączenia się do serwera. Zawiera pole \texttt{nick}, oznaczające wybrany przez klienta identyfikator.
 \item[newchannel] -- komunikat będący prośbą o założenie nowego kanału do rozmowy. Zawiera pole \texttt{name}, oznaczające wybraną przez klienta nazwę nowego kanału.
 \item[join] -- komunikat będący prośbą o przyłączenie do kanału. Zawiera pole \texttt{name}, oznaczające nazwę kanału do którego chce się przyłączyć klient.
 \item[exit] -- komunikat będący prośbą o wypisanie klienta z kanału rozmowy.
 \item[sendchannels] -- komunikat będący prośbą o przesłanie listy kanałów do których może dołączyć klient.
 \item[holepunch] -- komunikat żądający od serwera przekazania otwartego kanału komunikacji do użytkownika, którego identyfikator jest zawarty w treści komunikatu. \begin{large}TU COŚ TRZEBA ZMIENIĆ. \end{large}
 \item[emptyresponse] -- komunikat wysyłany w odpowiedzi na pytania, które \textit{de facto} są polecaniami i nie wymagają żadnej odpowiedzi, prócz potwierdzenia, że pakiet doszedł, np. \textbf{echorequest}, \textbf{channelcanceled}.
\end{description}


\subsection{Komunikaty wysyłane przez serwera do klienta}
\begin{description}
 \item[welcome] -- odpowiedź na \textbf{newclient}. Komunikat ten oznacza, że serwer zaakceptował wybrany identyfikator. Dla potwierdzenia znajduje się on w polu \texttt{nick}. Ponadto pakiet zawiera pole \texttt{channels}, będące listą nazw kanałów, do których może dołączyć klient. 
 \item[invalidnick] -- odpowiedź na \textbf{newclient}. Komunikat ten oznacza, że wybrany identyfikator nie został zaakceptowany. Zawiera pola \texttt{errid} oraz \texttt{desc} oznaczające numer błędu oraz jego opis.
 \item[echorequest] -- komunikat żądający od klienta potwierdzenia swojej obecności.
 \item[channellist] -- odpowiedź na \textbf{sendchannels}. W polu \texttt{channels} zawarta jest lista nazw kanałów, do których może dołączyć klient.
 \item[channelaccepted] -- odpowiedź na \textbf{newchannel}. Komunikat ten oznacza, że serwer akceptuje założenie przez klienta nowego kanału rozmowy.
  \item[channelrejected] -- odpowiedź na \textbf{newchannel}. Komunikat ten oznacza, że serwer odrzuca założenie przez klienta nowego kanału rozmowy. W treści komunikatu przesyłany jest numer błędu i jego opis.
 \item[joinaccepted] -- odpowiedź na \textbf{join}. Komunikat ten oznacza, że prośba klienta o dołączenie do pewnego kanału została zaakceptowana. W polu \texttt{name}, jako potwierdzenie, znajduje się nazwa kanału, do którego dołącza klient.
  \item[joinrejected] -- odpowiedź na \textbf{join}. Komunikat ten oznacza, że prośba klienta o dołączenie do pewnego kanału została odrzucona.Zawiera pola \texttt{errid} oraz \texttt{desc} oznaczające numer błędu oraz jego opis.
 \item[exitaccepted] -- odpowiedź na \textbf{exit}. Oznacza on, że klient został wypisany z kanału rozmowy. Ponadto w polu \texttt{channels} zawarta jest lista nazw kanałów, do których może dołączyć klient.
 \item[address] -- odpowiedź na \textbf{holepunch}. W treści komunikatu przesyłany jest identyfikator użytkownika wraz z jego adresem zewnętrznym i portem do komunikacji.
 \item[channelcanceled] -- komunikat informujący klienta, że kanał w którym się znajduje przestał istnieć.
  \item[userjoined] -- komunikat informujący klienta, że do kanału w którym się znajduje dołączył użytkownik. W treści komunikatu przesyłany jest identyfikator dołączającego użytkownika.
 \item[userleft] -- komunikat informujący klienta, że kanał w którym się znajduje opuścił użytkownik. W treści komunikatu przesyłany jest identyfikator odchodzącego użytkownika.
 \item[error] -- komunikat wysyłany gdy wystąpi bład inny niż te opisane wyżej. Zawiera pola \texttt{errid} oraz \texttt{desc} oznaczające numer błędu oraz jego opis.
\end{description}

\subsection{Komunikaty wysyłane prez klienta do klienta}
\begin{description}
\item[genericdata]  -- 
\end{description}


\end{document} 
