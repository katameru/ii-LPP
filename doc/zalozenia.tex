\documentclass[a4paper, 12pt]{article}
\usepackage{fancyhdr}
\usepackage[MeX]{polski}
\usepackage[utf8]{inputenc}
\usepackage{amsmath,amsthm,amssymb}
\usepackage{graphicx}
\usepackage{multirow}
\usepackage[lmargin=2.7cm]{geometry}
\usepackage{color}

\pagestyle{fancy}
\lhead{\textbf{Wtomigraj} -- Założenia ogólne}
\rhead{\bfseries}

\author{Marcin Januszkiewicz, Grzegorz Łoś}
\title{\textbf{Wtomigraj} -- Założenia ogólne}

\newcommand{\newtitle}[1]{
\begin{center}
\thispagestyle{empty}
{\Large Licencjacka Pracownia Oprogramowania, zespół 21}\\[0.5cm]
{\Large Instytut Informatyki Uniwersytetu Wrocławskiego 2011/12}\\[2.5cm]
{\huge Dokumentacja projektu \textbf{Wtomigraj}}\\[0.5cm]
{\huge #1}\\[0.5cm]
{\large Marcin Januszkiewicz, Grzegorz Łoś}\\[0.5cm]
\vfill
{\large Wrocław, \today}
\end{center}
\newpage
}
\renewcommand{\thesection}{\arabic{section}.}
\renewcommand{\thesubsection}{\arabic{section}.\arabic{subsection}.}
\renewcommand{\thesubsubsection}{\arabic{section}.\arabic{subsection}.\arabic{subsubsection}.}


\begin{document}

\newtitle{Założenia ogólne}

\break

\setcounter{page}{2}

\tableofcontents

\break

\section{Wstęp}

\subsection{Cel dokumentu}
Poniższy dokument opisuje ogólne założenia dotyczące platformy Wtomigraj. Przedstawiamy tu profil potencjalnych użytkowników oraz podstawowe przypadki użycia. W dokumencie są też zebrane istotne cechy platformy oraz stawiane przed nią wymagania. Szczegółowe informacje na temat architektury platformy i protokołu stosowanego przez platformę można znaleźć z załączonym dokumencie ,,Wtomigraj -- Opis architektury''.

\subsection{Dlaczego Wtomigraj}
Wśród użytkowników internetu można zaobserwować wysokie zainteresowanie grami przygodnymi (ang. ,,casual games''). Określa się tak gry, w których granie nie wymaga poświęcenia dużej ilości uwagi, czasu i wysiłku. Najczęściej są to gry które mogą być wyświetlane w przeglądarce, dzięki czemu odbiorca nie musi tracić czasu na instalację gry na komputerze a może uruchomić grę na przykład bezpośrednio z~Facebooka. Rynek tych gier jest rzeczywiście bardzo duży: jedna z największych firm, Zynga, ma roczne przychody rzędu 300 milionów dolarów. W odpowiedzi na oczekiwania internautów, w Sieci znalazło się wiele portali z grami, jednak większość z~nich oferuje jedynie gry jednoosobowe. Tymczasem dużą popularnością cieszą się gry wieloosobowe, w których gracze mogą rywalizować nie tylko ze sztuczną inteligencją, ale też z innymi graczami. Wtomigraj wychodzi naprzeciw temu zapotrzebowaniu przedstawiając wygodny interfejs programowania aplikacji oraz gotowe serwery mające na celu ułatwienie tworzenia gier komunikujących się przez internet.

\section[Opis użytkownika]{Opis użytkownika}

\subsection[Opis użytkowników]{Opis użytkowników}
Użytkownikami platformy Wtomigraj są programiści, którzy będą korzystać z gotowych funkcji i programów do tworzenia własnych gier sieciowych. Dzięki naszej platformie będą mogli skupić się na implementowaniu funkcjonalości gry bez konieczności martwienia się o szczegóły komunikacji przez internet. 

\subsection[Dane statystyczne dotyczące użytkowników i rynku]{Dane statystyczne dotyczące użytkowników i rynku}
Jak podaje portal http://www.tiobe.com język Java, w którym napisana jest nasza platforma jest obecnie drugim co do popularności językiem programowania. Oznacza to, że liczba programistów, którzy mogą zainteresować się naszą platformą jest bardzo duża.

\subsection[Podstawowe potrzeby użytkownika]{Podstawowe potrzeby użytkownika}
Podstawową potrzebą użytkowników platformy jest otrzymanie prostej w użyciu biblioteki programistycznej, dzięki której nie będą musieli samemu programować komunikacji między dwoma komputerami przez internet. Biblioteka powinna być prosta, tak by nie trzeba było wiele czasu poświęcać nauce jej używania, a także elastyczna, tak by programista nie był ograniczony przy tworzeniu własnej gry.

\section[Ogólny opis Wtomigraja]{Ogólny opis Wtomigraja}
Wtomigraj jest platformą programistyczną, służącą tworzeniu aplikacji internetowych, które umożliwią skomunikowanie się dwóch lub więcej komputerów w Internecie, w celu odbycia wspólnej gry.

\subsection[Schemat Wtomigraja]{Schemat Wtomigraja}
Wtomigraj składa się z dwóch części. Pierwszą jest serwer, który pośredniczy w nawiązaniu połączenia między aplikacjami graczy. Drugą częścią jest implementacja klienta mogącego komunikować się z tym serwerem. Pozwala ona zmniejszyć narzut programistyczny na wykorzystanie możliwości serwera poprzez dostarczenie odpowiednich klas i funkcji do komunikacji z serwerem. Programista gry nie musi na przykład wnikać w szczegóły wykorzystywanego protokołu, musi jedynie umieć wywołać odpowiednie metody biblioteczne. 

\subsection[Podsumowanie możliwości]{Podsumowanie możliwości}
Platforma Wtomigraj jest olbrzymim ułatwieniem dla programisty, gdyż:
\begin{itemize}
 \item niesie ze sobą gotowy program serwera, który może obsługiwać powstające gry,
 \item ma zaimplementowany protokół komunikacji na liniach klient-serwer oraz klient-klient,
 \item daje programiście wygodny interfejs do wysyłania danych pomiędzy klientami na różnych komputerach,
 \item nie wnosi niemal żadnych ograniczeń na rodzaj gier jakie mogą powstawć przy użyciu tej platformy.
\end{itemize}

\subsection[Założenia i zależności]{Założenia i zależności}
Aby kompilować kod oparty na platformie Wtomigraj potrzebne jest JDK w co najmniej w wersji 5. Ponadto platforma Wtomigraj korzysta z biblioteki JSON, którą pobrać będzie musiał także programista gry.

\section[Cechy Wtomigraja]{Cechy Wtomigraja}
\paragraph{Serwer obsługuje wiele gier.} Dzięki temu wystarczy uruchomić jeden serwer, aby równoczesnie obsługiwać, np. szachistów i brydżystów.
\paragraph{Dodawanie nowych gier do serwera.} Wystarczy wprowadzić zmiany w pliku konfiguracyjnym.
\paragraph{Komunikacja klient-serwer.} Jest całkowicie obsługiwana przez bibliotekę i użytkownik naszej platformy nie musi się nią martwić.
\paragraph{Komunikacja klient-klient.} Realizowane jest nawiązywanie połączenia między klientami, a także wysyłanie pakietów w formacie JSON pomiędzy nimi. Programiście gry pozostaje zaimplementować zapis i odczyt danych gry z pakietu JSON.

\section[Przykładowe przypadki użycia]{Przykładowe przypadki użycia}
\subsection{Stworzenie serwisu z grami} Firma tworzy gry wieloosobowe przy użyciu naszej platformy i umieszcza je w swoim serwisie. Komunikacja pomiędzy wszystkimi grami jest obsługiwana przez jeden serwer wspólny dla wszystkich gier co ogranicza koszty tworzenia i wdrażania osobnych serwerów dla każdej gry. Korzystanie z jednolitego interfejsu ułatwia tworzenie nowych gier.
\subsection{Pisanie własnej gry.} Programista tworzy grę korzystając z naszej platformy. Oszczędza dzięki temu czas który może spędzić szlifując szczegóły gry. Serwer rozprowadzany jest razem z grą, dzięki czemu nie jest konieczny centralny serwis aby grać przez sieć. Można nawet grać przez sieć lokalną, bez Internetu.

\section{Inne wymagania}
\subsection{Licencjonowanie i instalacja.}
Kod udostępniony jest na licencji GNU GPL. Dzięki temu programiści korzystają z~naszej platformy bezpłatnie, jednak powstające oprogramowanie również musi być darmowo dostępne dla użytkowników końcowych. Liczymy, że dzięki takiej licencji zyska ona na popularności. Aby korzystać z platformy Wtomigraj wystarczy pobrać pakiet wtomigraj.jar. 


\subsection{Wymagania efektywnościowe}
Biblioteka musi umożliwiać szybką komunikację dlatego tworzymy projekt korzystając z protokołu UDP, a nie TCP, który choć jest wygodniejszy wiąże się z większym narzutem czasowym.

\subsection[Wymagania dokumentacyjne]{Wymagania dokumentacyjne}
Kod źródłowy przechowywany będzie w repozytorium GIT. Wraz z nim będzie znajdować się aktualna wersja dokumentacji, zarówno jako komentarze w kodzie źródłowym, a także jako osobny dokument, opis całej architektury systemu. Ponadto każdy commit do repozytorium obowiązkowo musi zawierać opis zmian.

\end{document} 