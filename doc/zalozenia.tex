\documentclass[11pt,leqno]{article}
\usepackage[polish]{babel}
\usepackage[utf8]{inputenc}
\usepackage{setspace}
\usepackage[T1]{fontenc}
\frenchspacing
\newcommand{\newtitle}[1]{
\begin{center}
\thispagestyle{empty}
{\Large Licencjacka Pracownia Oprogramowania, zespół 12}\\[0.5cm]
{\Large Instytut Informatyki Uniwersytetu Wrocławskiego 2011/12}\\[2.5cm]
{\huge Dokumentacja projektu \textbf{Wtomigraj}}\\[0.5cm]
{\huge #1}\\[0.5cm]
{\large Marcin Januszkiewicz, Grzegorz Łoś}\\[0.5cm]
\vfill
{\large Wrocław, \today}
\end{center}
\newpage
}
\renewcommand{\thesection}{\arabic{section}.}
\renewcommand{\thesubsection}{\arabic{section}.\arabic{subsection}.}
\renewcommand{\thesubsubsection}{\arabic{section}.\arabic{subsection}.\arabic{subsubsection}.}


\begin{document}
\newtitle{Wstęp}


\newpage
\section{Wstęp}

\subsection{Cel dokumentu}
W dokumencie sa zebrane, zdefiniowane i analizowane ogólne potrzeby uzytkownika oraz cechy produktu. Dokładne informacje o portalu zostaną opisane w kolejnych dokumentach. 

\subsection{Ogólny opis produktu}
Wśród użytkowników internetu można zaobserwować wysokie zapotrzebowanie na prostą rozrywkę, na przykład w postaci gier internetowych. Zazwyczaj wykorzystywane są one, aby zrobić sobie chwilę przerwy, pomagają na kilka minut odetchnąć od wykonywanej pracy. Zaletą w stosunku do gier ze sklepowych półek są intuicyjne sterowanie i nieskomplikowane zasady, a~także darmowa dostępność oraz fakt, że gry internetowe nie wymagają żadnej instalacji. W odpowiedzi na oczekiwania internautów, w sieci znalazło się wiele portali z grami. Jednak większość z nich oferuje jedynie gry dla pojedynczego gracza. Tymczasem rosnącą popularnośćią cieszą się gry wieloosobowe, w~których gracze mogą ze sobą rywalizować. Portal Wtomigraj wychodzi naprzeciw temu zapotrzebowaniu przedstawiając swoim użytkownikom ofertę gier wieloosobowych. 

\end{document}
